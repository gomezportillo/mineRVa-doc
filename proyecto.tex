\documentclass[a4paper,11pt]{book}
% Long footnote URLS
\usepackage[hyphens]{url}

\usepackage{listings}
\usepackage{xspace}
\usepackage[utf8]{inputenc}
\usepackage[spanish]{babel}
\usepackage{pdfpages}
\usepackage{amsmath}
\usepackage[official]{eurosym}

\decimalpoint
\usepackage{dcolumn}
\newcolumntype{.}{D{.}{\esperiod}{-1}}
\makeatletter
\addto\shorthandsspanish{\let\esperiod\es@period@code}
\makeatother

\RequirePackage{verbatim}
\usepackage{fancyhdr}
\usepackage{graphics, graphicx, float}
\usepackage{afterpage}

\usepackage{longtable}

\usepackage[pdfborder={0 0 0}]{hyperref}
\usepackage[printonlyused]{acronym}
\newcommand{\Acro}[2]{\acro{#1}{#2}\acused{#1}}

% ********************************************************************
% Re-usable information
% ********************************************************************
\newcommand{\myTitle}{MineRVa\xspace}
\newcommand{\mySubtitle}{Experiencia de juego en Realidad Virtual en un museo\xspace}
\newcommand{\myEnglishSubtitle}{Virtual reality game experience in a museum\xspace}
\newcommand{\myDegree}{Máster en Ingeniería Informática\xspace}
\newcommand{\myName}{Pedro Manuel Gómez-Portillo López\xspace}
\newcommand{\myEmail}{pedromanuel.gomezportillo@gmail.com\xspace}
\newcommand{\myWebsite}{https://github.com/gomezportillo/mineRVa\xspace}
\newcommand{\myProf}{Francisco Luis Gutiérrez Vela\xspace}
\newcommand{\myFaculty}{Escuela Técnica Superior de Ingenierías Informática y de Telecomunicación\xspace}
\newcommand{\myFacultyShort}{E.T.S. de Ingenierías Informática y de Telecomunicación\xspace}
\newcommand{\myDepartment}{Departamento de Lenguajes y Sistemas Informáticos\xspace}
\newcommand{\myUni}{\protect{Universidad de Granada}\xspace}
\newcommand{\myLocation}{Granada\xspace}
\newcommand{\myTime}{\today\xspace}
\newcommand{\myPalabrasclave}{Realidad virtual, desarrollo de videojuegos, Unity, VRTK}
\newcommand{\myKeywords}{Virtual reality, game development, Unity, VRTK}


\hypersetup{
pdfauthor = {\myName (gomezportillo@correo.ugr.es)},
pdftitle = {\myTitle},
pdfsubject = {},
pdfkeywords = {\myKeywords},
pdfproducer = {pdflatex}
}

\usepackage{colortbl,longtable}
\usepackage[stable]{footmisc}

\pagestyle{fancy}
\fancyhf{}
\fancyhead[LO]{\leftmark}
\fancyhead[RE]{\rightmark}
\fancyhead[RO,LE]{\textbf{\thepage}}
\renewcommand{\chaptermark}[1]{\markboth{\textbf{#1}}{}}
\renewcommand{\sectionmark}[1]{\markright{\textbf{\thesection. #1}}}

\setlength{\headheight}{1.5\headheight}

\newcommand*\justify{
	\fontdimen2\font=0.4em
	\fontdimen3\font=0.2em
	\fontdimen4\font=0.1em
	\fontdimen7\font=0.1em
	\hyphenchar\font=`\-
}
\newcommand{\HRule}{\rule{\linewidth}{0.5mm}}

\newcommand{\dedication}[1]{%
  \cleardoublepage
  \thispagestyle{empty}
  \null\vspace{\stretch{1}}
  \begin{flushright}
    \textit{#1}
  \end{flushright}
  \vspace{\stretch{2}}\null
  \cleardoublepage
}

\definecolor{gray97}{gray}{.97}
\definecolor{gray75}{gray}{.75}
\definecolor{gray45}{gray}{.45}
\definecolor{gray30}{gray}{.94}

\lstset{ frame=Ltb,
     framerule=0.5pt,
     aboveskip=0.5cm,
     framextopmargin=3pt,
     framexbottommargin=3pt,
     framexleftmargin=0.1cm,
     framesep=0pt,
     rulesep=.4pt,
     backgroundcolor=\color{gray97},
     rulesepcolor=\color{black},
     stringstyle=\ttfamily,
     showstringspaces = false,
     basicstyle=\scriptsize\ttfamily,
     commentstyle=\color{gray45},
     keywordstyle=\bfseries,
     numbers=left,
     numbersep=6pt,
     numberstyle=\tiny,
     numberfirstline = false,
     breaklines=true,
}

\lstnewenvironment{listing}[1][]
   {\lstset{#1}\pagebreak[0]}{\pagebreak[0]}

\lstdefinestyle{C} {
	basicstyle=\scriptsize,
	frame=single,
	language=C,
	numbers=left
}

\lstdefinestyle{C++} {
	basicstyle=\small,
	frame=single,
	backgroundcolor=\color{gray30},
	language=C++,
	numbers=left
}

\lstdefinestyle{Consola} {
   	basicstyle=\scriptsize\bf\ttfamily,
    backgroundcolor=\color{gray30},
    frame=single,
    language=shell,
    numbers=none
}

\lstdefinestyle{XML} {
	basicstyle=\scriptsize,
	frame=single,
	language=XML,
	numbers=left
}

\newcommand{\bigrule}{\titlerule[0.5mm]}

\makeatletter
\def\clearpage{%
  \ifvmode
    \ifnum \@dbltopnum =\m@ne
      \ifdim \pagetotal <\topskip
        \hbox{}
      \fi
    \fi
  \fi
  \newpage
  \thispagestyle{empty}
  \write\m@ne{}
  \vbox{}
  \penalty -\@Mi
}
\makeatother

\usepackage{pdfpages}


%%------------- ESTILOS PROPIOS ---------------------------
\usepackage{caption}
\captionsetup{justification=centering}

\newcommand{\MineRVa}{\textit{MineRVa} }
\renewcommand*{\lstlistlistingname}{Índice de listados}
\renewcommand{\listtablename}{Índice de tablas} %% not working 

\addto\captionsenglish{\renewcommand{\appendixname}{Apéndices}}

\newcommand{\appendixtitle} {
  \cleardoublepage
%  \thispagestyle{empty}
  \vspace*{5cm}
  \begin{center}
    \sffamily\scshape\Large\scalebox{3}{APÉNDICES}
  \end{center}
}

\usepackage{csquotes}
\usepackage{eurosym}

%% -- letras capitales
\usepackage{lettrine}
\newcommand{\drop}[2]{
  \lettrine[lines=2]
  {\textcolor[gray]{0.4}{\textbf{#1}}}{#2}
  }

% Cajas grises
\usepackage{tcolorbox}

%% -------------- Código C#
\definecolor{bluekeywords}{rgb}{0,0,1}
\definecolor{greencomments}{rgb}{0,0.5,0}
\definecolor{redstrings}{rgb}{0.64,0.08,0.08}
\definecolor{xmlcomments}{rgb}{0.5,0.5,0.5}
\definecolor{types}{rgb}{0.17,0.57,0.68}

\usepackage{listings}
\lstset{language=[Sharp]C,
captionpos=b,
%numbers=left, %Nummerierung
%numberstyle=\tiny, % kleine Zeilennummern
frame=lines, % Oberhalb und unterhalb des Listings ist eine Linie
showspaces=false,
showtabs=false,
breaklines=true,
showstringspaces=false,
breakatwhitespace=true,
escapeinside={(*@}{@*)},
commentstyle=\color{greencomments},
morekeywords={partial, var, value, get, set, Transform, GameObject, endif},
keywordstyle=\color{bluekeywords},
stringstyle=\color{redstrings},
basicstyle=\ttfamily\small,
}


%% -------------------- COPYRIGHT ------------------------------
\usepackage{setspace}

\newcommand{\myLicense}{
  \begin{minipage}{1.1\textwidth}
    \begin{singlespace}
      
      \smallskip
        Este documento y la información que contiene se proporcionan de forma confidencial, con el único fin de que el destinatario del título los lea, y no puede ser divulgado a ningún tercero ni utilizado para ningún otro propósito sin el permiso expreso por escrito del autor.

  \end{singlespace}
  \end{minipage}
}

\newcommand{\copyrightpage}{
  \newpage
  \begin{singlespace}
    \null \vfill \noindent
    \textbf{\myName} \par
    \smallskip \noindent
    \myLocation\ -- Spain

    \vspace{-0.65cm}
    \begin{tabbing}
      \hspace*{1.7cm} \= \\
      \emph{E-mail:}   \> \myEmail \\
      \emph{Web site:} \> \url{https://github.com/gomezportillo/mineRVa} \\
    \end{tabbing}
    \vspace{-0.65cm}

    \noindent
    \copyright\ 2019 \myName \par
    \smallskip \noindent
    \begin{minipage}{0.8\textwidth} \raggedright \footnotesize
      \myLicense
    \end{minipage}
  \end{singlespace}
  \cleardoublepage
}

\begin{document}
\input{portada/portada}
\chapter*{}

\input{portada/contraportada}

\cleardoublepage
\thispagestyle{empty}

\copyrightpage

\begin{center}
{\large\bfseries \myTitle: \mySubtitle}\\
\end{center}

\begin{center}
\myName \\
\end{center}

\vspace{0.7cm}
\noindent{\textbf{Palabras clave}: \myPalabrasclave }\\

\vspace{0.7cm}
\noindent{\textbf{Resumen}}\\

% Resumen
Las tecnologías asociada a la Realidad Virtual permiten que personas que no pueden acceder a centros, como puede ser un museo, puedan visitarlo de una forma más o menos realista y en muchos casos incluso motivadora por la tecnología que se emplea.

\thispagestyle{empty}

\cleardoublepage

\begin{center}
	{\large\bfseries \myTitle: \myEnglishSubtitle}\\
\end{center}

\begin{center}
	\myName \\
\end{center}

\vspace{0.7cm}
\noindent{\textbf{Keywords}: \myKeywords }\\

\vspace{0.7cm}
\noindent{\textbf{Abstract}}\\

% Abstract




% \chapter*{}
% \thispagestyle{empty}

% \noindent\rule[-1ex]{\textwidth}{2pt}\\[4.5ex]

% Yo, \textbf{\myName}, alumno de la titulación \myDegree de la \textbf{\myFaculty}, con DNI 71722388Q, autorizo la ubicación de la siguiente copia de mi Trabajo Fin de Máster en la biblioteca del centro para que pueda ser consultada por las personas que lo deseen.

% \vspace{6cm}

% \noindent Fdo: \myName

% \vspace{2cm}

% \begin{flushright}
% \myLocation a \myTime.
% \end{flushright}


\chapter*{}
\thispagestyle{empty}

\noindent\rule[-1ex]{\textwidth}{2pt}\\[4.5ex]

D. \textbf{\myProf}, Profesor del \myDepartment de la \myUni.

\vspace{0.5cm}

\textbf{Informa:}

\vspace{0.5cm}

Que el presente trabajo, titulado \textit{\textbf{\myTitle, \mySubtitle}}, ha sido realizado bajo su supervisión por \textbf{\myName}, y autoriza la defensa de dicho trabajo ante el tribunal que corresponda.

\vspace{0.5cm}

Y para que conste, expide y firma el presente informe en \myLocation a \myTime.

\vspace{1cm}

\textbf{El director:}

\vspace{5cm}

\noindent \textbf{\myProf}


\dedication{A mi historiadora del arte}

\frontmatter
\tableofcontents
\listoffigures
\listoftables
\lstlistoflistings
\chapter{Lista de acrónimos}
{
\small
\begin{acronym}[XXXXXXXX]

\acro{IDE}     {Integrated Development Environment}
\acro{TFM}     {Trabajo Final de Máster}
\acro{VR}      {Virtual Reality}
\acro{VRTK}    {Virtual Reality Toolkit}

\acro{CSS}     {Cascading Style Sheets}
\acro{FPS}     {Frames Per Second}
\acro{GPU}     {Graphical Processing Unit}
\acro{GUI}     {Graphical User Interface}  
\acro{HCI}     {Human-Computer Interaction}
\acro{HTML5}   {Hypertext Markup Language, version 5}
\acro{HTTP}    {Hypertext Transfer Protocol}  
\acro{JPEG}    {Joint Photographic Experts Group}
\acro{JS}      {JavaScript}
\acro{KISS}    {Keep it simple, stupid!} 
\acro{RTS}     {Real Time System}
\acro{RUP}     {Rational Unified Process}
\acro{SDK}     {Software Development Kit}
\acro{TCP}     {Transmission Control Protocol}  
\acro{UPD}     {Unified Process Development}

\end{acronym}
}


% \ac{OO}   la primera vez \acf, después \acs
% \acs{OO}  short: OO
% \acf{OO}  full : Object Oriented (OO)
% \acl{OO}  large: Object Oriented
% \acx{OO}         OO (Object Oriented)

% usa \Acro cuando no debe aparecer nunca expandido en el texto

% Local variables:
%   TeX-master: "main.tex"
% End:


\mainmatter
\setlength{\parskip}{8pt}

\chapter{Introducción}
\label{chap:introduccion}

industria del ocio

el numero de visitantes de museos está creciendo

\url{https://www.abc.es/cultura/arte/abci-museos-espanoles-crecen-2018-201901021635_noticia.html}

la industria del videojuego ha crecído muchisimo y genera mucho dinero

grafica

nuevas tecnicas de interaccion con los usuarios

VR


\section{Motivación}

estoy muy interesado el desarrollo de videojuegos. hecho estos cursos y estos juegos...

tb me interesa el diseño 3d. blender desde hace años...

\section{Entorno}

\section{Impacto socio-económico}

qué se pretende a nivel social/economico conseguir con este proyecto

\section{Objetivos}

\subsection{Objetivo principal}

El principal objetivo de \MineRVa es 

\subsection{Objetivos secundarios}

\section{Estructura del documento}
\chapter{Estado del arte}
\label{chap:estado_arte}

\chapter{Análisis inicial del problema}
\label{chap:analisis_problema}

\chapter{Tecnología a usar}
\label{chap:tecnologia}

\chapter{Metodologías a usar en este proyecto}
\label{chap:metodologias}

Es importante seguir los principios y buenas prácticas de la ingeniería del software...

\section{Metodología de trabajo}

Diferencias entre proceso de desarrollo y metodología de desarrollo

\subsection{Proceso de desarrollo}

Agile Software Dev.

Sus 4 principios básicos 

\subsection{Metodología de desarrollo}

SCRUM

sus 3 pilares (transparencia, inspecion, adaptacion)

Describirlo? artefactos, eventos, workflow, roles...
\chapter{Plan de entregas}
\label{chap:plan_entregas}

\chapter{Desarrollo}
\label{chap:desarrollo}

\drop{C}{  omo} ya se ha comentado antes, el desarrollo de este proyecto ha supuesto un reto importante ya que las tecnologías utilizadas eran totalmente desconocidas. En este capítulo se explica el proceso seguido a lo largo de su desarrollo, además de presentar los principales problemas encontrados y las soluciones con las que se han abordado.

Por otro lado, para cada entrega se incluirán los bocetos realizados para los modelos, además del modelo renderizado en Unity, y uno de los fragmentos de código desarrollado que se considere más interesante, además de los diagramas pertinentes que ayuden a entenderlo.

Además, como algunos de los elementos desarrollados en una entrega sufrieron cambios en las siguientes, solo se explicará la versión final, por lo que puede que algo de lo explicado para una entrega no corresponda exactamente con lo enseñado en el vídeo de la misma.

\section{Entrega 0}

El objetivo de esta primera entrega fue, por un lado, establecer la estructura principal del proyecto y esbozar una primera versión del a narrativa y por otro realizar una primera toma de contacto con el desarrollo de Unity, las tecnologías \acs{VR} e instalar y configurar el entorno de desarrollo para poder empezar a trabajar en la siguiente entrega.

\subsection{Estructuración del proyecto}

El primer paso lógico tras proponer al tutor de este \acs{TFM} el proyecto y ser aprobado fue empezar a definirlo. Para ello, y partiendo de la idea principal de desarrollar una experiencia de juego haciendo uso de tecnologías \acs{VR} que pusiera en contacto con el mundo del arte a personas que suelen y no visitar museos, se generó el documento que puede verse en el anexo \ref{anexo:guia-salas}. 

Este documento comienza a detallar la narrativa del juego y la integra en una visita por un museo ficticio y, aunque terminó por sufrir varios cambios importantes, sirvió para definir el punto de partida del proyecto.

A lo largo de este documento se intenta crear una historia interesante para el jugador al mismo tiempo que generar un museo ficticio pero realista y coherente en el que se pueda desarrollar dicha narrativa. Para ello, se ha seguido el orden cronológico por las épocas más importantes en la historial de arte, definiendo una sala con una historia diferente para cada una.

A la hora de elegir los cuadros que se mostrarían en las salas se intentó buscar aquellos más representativos de su época. Para ello, se contó con el asesoramiento de una historiadora del arte, que ha sido quien ha dado el visto bueno al rigor artístico del museo y sus salas.

\subsection{Configuración del entorno de desarrollo}

Antes de poder empezar a desarrollar el proyecto fue necesario elegir un \acs{IDE} y un framework con el que trabajar. Como ya se ha explicado en los capítulos \ref{chap:estado_arte} y \ref{chap:tecnologia}, tras comparar las ventajas y desventajas de los entornos de desarrollo y librerías disponibles en el mercado se decidió trabajar con Unity, el framework \acs{VRTK} y la librería SteamVR para lo que, tras instalarlas, hubo que importarlas manualmente al proyecto de Unity.

\subsection{Primera toma de contacto con Unity}

A continuación se presenta un resumen de la interfaz de Unity para que el lector pueda entender algunos de los conceptos de los que se hablará más adelante.

\begin{figure}[!h]
\begin{center}
\includegraphics[width=1\textwidth]{imagenes/7/interfaz-unity.png}
\caption{Resumen de la interfaz de Unity}
\label{fig:interfaz-unity}
\end{center}
\end{figure}

\begin{enumerate}
    \item Vista de la escena actual, en la que el usuario puede obtener una vista previa de la escena e interactuar con los objetos tridimensionales para colocarlos. Funciona de manera parecida a Blender.
    
    \item Vista de la jerarquía de la escena, en la que pueden verse los objetos que hay y sus relaciones; por ejemplo, si están emparentados.
    
    \item Vista del inspector en la que aparece la información de los componentes de un objeto seleccionado. Un componente puede ser prácticamente cualquier cosa desde un script a un material.
    
    \item Vista del proyecto, donde aparecen todas las carpetas disponibles.
    
    \item Vista donde aparecen los elementos de la carpeta seleccionada. En este caso, pueden verse algunos de los scripts con los que se ha trabajado.
    
    \item Consola de salida en la que aparece información del proyecto.
\end{enumerate}

\section{Entrega 1}

Como se indicó en el capítulo \ref{chap:plan_entregas}, el objetivo final de las iteraciones de esta entrega fue aprender a importar modelos a Unity desde Blender y diseñar e implementar el tutorial del proyecto y que éste fuera completamente funcional. 

\subsection{Modelado e importación}

Lo primero que se hizo antes de comenzar a modelar en Blender, ya que es mucho menos productivo empezar a trabajar sin una idea previa, fue diseñar un boceto en papel en el poder basar el modelado posterior.

Como se consideró que el museo sería más realista si en lugar de empezar directamente en él el jugador tuviera que recorrer un pequeño pasillo que funcionara de antesala y desde el que se pudiera ver el exterior, fue el primer boceto que se hizo, y tras él se dibujó la sala que actuaría de tutorial. Esta sala tendría que presentar un cuadro muy reconocible y una pequeña prueba relacionada con él, por lo que se decidió utilizar el cuadro \textit{El Hijo del Hombre} de René Magritte (1964) y que el jugador tuviera que cambiar de sitio una pieza de fruta relacionada con este cuadro, de este modo aprendiendo que puede interaccionar con los elementos virtuales y que habrá relación entre las pruebas y las obras de arte de tal manera que el reto sea darse cuenta de estas ideas y no la prueba en sí. La imagen \ref{fig:bocetos-salas-0-1} muestra el boceto de estas dos salas, que fue dibujado antes de comenzar a modelarlas.

\begin{figure}[!h]
\begin{center}
\includegraphics[width=.8\textwidth]{imagenes/7/bocetos/boceto-sala-0-1.png}
\caption{Boceto de la antesala y la sala de tutorial}
\label{fig:bocetos-salas-0-1}
\end{center}
\end{figure}

Una vez terminados los bocetos, se modeló la primera sala en Blender se comenzó a trabajar en importarla desde Unity, para lo que es necesario crear una Escena e importar en ella el archivo Blender desde el gestor de archivos. Una vez que lo hagamos, aparecerá como un \textbf{Prefab}\footnote{\url{https://docs.unity3d.com/es/current/Manual/Prefabs.html}}. De este modo, cuando el archivo Blender se modifique Unity lo detectará y actualizará su copia local, aunque por ser un Prefab no pueden modificarse desde Unity sin perder esta propiedad.

Actualmente, Blender cuenta con dos motores de renderizado; Blender Internal y Blender Cycles, y no son compatibles entre sí. Esto quiere decir que si por ejemplo creamos un material en Blender Internal y luego cambiamos a Cycles, éste no aparecerá. Aunque en teoría Unity trabaja mejor con los materiales de Blender Internal, al importarlos no aparecen como deberían y no pueden modificarse sus propiedades como su color, su \textit{metalicidad} o su mapa de normales, por lo que ha sido necesario rehacer todos los materiales de los modelos y volver a aplicarlos a mano.

Vamos a tomar como ejemplo una de las paredes de ladrillos para ver el flujo de trabajo de los materiales; tras modelarla, habría que descargar una textura para ella, para lo que se ha utilizado el sitio web \url{https://3dtextures.me/} que proporciona texturas procedurales gratuitas y con mapas de normales y rugosidad con licencia libre. Una vez hecho, se crea un material en Blender al que se le aplica la textura para ver cómo quedaría, aunque Blender Internal no da la opción de añadir más mapas a la textura. Tras esto, se importa el modelo desde Unity y se crea un nuevo material, con la misma textura, al que se le añaden y configuran el resto de mapas.

La imagen \ref{fig:unity-sala-0} muestra el resultado final del modelado y la importación a Unity. Como el exterior con árboles, que puede verse a través de las ventanas, se reutiliza en otras salas se ha movido a una escena aparte que se importa cuando es necesario, recudiendo de este modo el peso de los modelos.

\begin{figure}[!h]
\begin{center}
\includegraphics[width=0.85\textwidth]{imagenes/7/salas-unity/unity-sala-0.png}
\caption{Sala 0 renderizada en Unity}
\label{fig:unity-sala-0}
\end{center}
\end{figure}

Tras ello se hizo lo mismo con la sala de tutorial, que puede verse en la figura \ref{fig:unity-sala-1}. Para dar a entender al jugador que están relacionadas, se ha utilizado la misma textura de ladrillos para las paredes y el mismo techo de cristal, lo que ayuda a aumentar la claridad y dar una sensación de amplitud,

\begin{figure}[!h]
\begin{center}
\includegraphics[width=0.85\textwidth]{imagenes/7/salas-unity/unity-sala-1.png}
\caption{Sala 1 renderizada en Unity}
\label{fig:unity-sala-1}
\end{center}
\end{figure}

Además, aunque las cajas de colisiones automáticas de Unity funcionan bien para objetos no lo hacen para habitaciones, ya que estas cajas la rodean y no permiten detectar colisiones interiores. Por ello, ha sido necesario definir manualmente estas colisiones, para lo que se han utilizado los componentes \texttt{Box Collider} para cada una de la paredes, el techo y el suelo. Estas cajas de colisiones definen los \textit{limites físicos} de los objetos e impiden que el jugador los atraviese.

\subsection{Viajar entre salas}

Una escena en Unity es una unidad que permite incluir en ella elementos que estén estrechamente relacionados desde el punto de vista del juego, como modelos o scripts. Desde la documentación de Unity se anima al desarrollador a encapsular cada nivel del juego en una escena; así, la equivalencia que se ha usado en este proyecto es de una sala por escena.

Una vez que las dos salas estaban modeladas se trabajó en hacer que el jugador pudiera viajar entre ellas; para ello, se escribió un script en C\# que permite cambiar la escena actual a otra cuando el jugador toque una puerta 

Para ello, lo primero que se hizo fue añadir una caja de colisiones sin físicas a la puerta, lo que permite añadir un \textit{listener} para detectar colisiones y poder activar otras funciones. Tras ello, se desarrolló y añadió un script como componente, que puede verse simplificado en el listado \ref{lst:viajar-salas}, que usa la clase \texttt{SceneManager} para cambiar la escena cuando el jugador colisiona con ella tras comprobar previamente que no se encuentra cargada. En él se declaran dos variables públicas para poder definirlas desde el propio inspector de Unity más cómodamente, como puede verse en la figura \ref{fig:door-teleporter-inspector}, lo que añade flexibilidad y reutilización al código. El script entero se encuentra en el archivo \texttt{DoorTeletransporter.cs}.

\begin{lstlisting}[caption=Fragmento del script para viajar entre salas, label=lst:viajar-salas]
public string scene_name;
public float fadingTime = 10.0f;

private void OnTriggerEnter(Collider other)
{
    if (scene_name != "" && !SceneManager.GetSceneByName(scene_name).isLoaded)
    {
        SceneManager.LoadScene(scene_name, LoadSceneMode.Single);
    }
}
\end{lstlisting}

\begin{figure}[!h]
\begin{center}
\includegraphics[width=0.6\textwidth]{imagenes/7/door-teleporter-inspector.jpg}
\caption{Script para cambiar de salas desde el inspector}
\label{fig:door-teleporter-inspector}
\end{center}
\end{figure}

Además, Unity utiliza la convención \textit{CamelCase} para nombrar sus variables, que es un estilo de escritura que se aplica a frases que omite los espacios y hace que cada palabra empiece en mayúsculas . Por ello, si la implementamos en nuestro código es capaz de reconocerlo y representar el nombre de las variables en el inspector correctamente.

Por otro lado, cada script puede implementar dos funciones, \texttt{Start()} y \texttt{Update()}, que se ejecutan automáticamente al inicio de la escena y en cada frame, respectivamente. Este paradigma es totalmente distinto a la programación lineal, ya que es necesario realizar cualquier cambio de manera iterativa; por ejemplo, si queremos mover un objeto un metro durante un segundo, deberemos parametrizar las distancias y los tiempos para que esta posición se actualice, aproximadamente, sesenta veces cada segundo.

\subsection{Interacción con objetos virtuales}

Una vez modeladas las dos salas, se comenzó a trabajar en hacer que el jugador pudiera interactuar con los objetos virtuales. Al estar utilizando el framework \acs{VRTK} se han podido hacer uso de sus funciones para facilitar mucho el trabajo.

Lo primero que se hizo, tras modelar las tres piezas de fruta (una manzana, un plátano y una pera) e importarlas a Unity fue dotarlas de físicas, para lo que se utilizaron los componentes \texttt{Box Collider} para definir sus límites y \texttt{Rididbody} para hacerlas responder a la gravedad. Tras ello se hizo que interactuaran con los mandos del jugador con ayuda de los componentes \texttt{VRTK\_Interactable\_Object}, \texttt{VRTK\_Child\_Of\_Controller}, \texttt{VRTK\_Interact\_Haptics} y se hizo que apareciera un borde amarillo cuando su caja de colisión detectara el mando con ayuda del componente \texttt{VRTK\_-} \texttt{Outline\_Object\_Highlighter}. Tras ello, se utilizó una \textit{snap drop zone} o zona en la que poder colocar objetos, para lo que se adaptó uno de los modelos proporcionados por el framework. Tras configurarlo adecuadamente, fue posible colocar sobre esta zona piezas de fruta y que estas automáticamente adquirieran la posición y rotación adecuadas.

Lo que se explica a continuación no se hizo hasta dos entregas posteriores, pero se contará ahora por estar estrechamente relacionado con ello. Para evitar que el jugador saliese de la sala sin hacer caso al vigilante, se añadió un script a la \textit{snap drop zone} antes mencionada que detectara cuando el jugador dejaba un objeto con la etiqueta \texttt{Apple} para desbloquear la puerta. Por un lado, para conseguir detectar cuando el jugador colocaba objetos en la zona se añadió un \textit{listener} al evento \texttt{ObjectSnappedToDropZone}, emitido por el componente \texttt{VRTK\_SnapDropZone}, que permite saber programáticamente cuándo ocurre esto. Por otro, para desabilitar la puerta y evitar que el jugador viaje entre salas, se desabilita por defecto su caja de colisiones y solo se vuelve a activar cuando el jugador coloca la manzana en su sitio. Más adelante también se utilizaría este sistema para activar uno de los diálogos del vigilante que felicitase al jugador por haberlo hecho bien.

Como resultado de esta entrega se generó el primer entregable y, por tanto, se grabó un vídeo presentando el proyecto y enseñando los avances, que puede verse en el siguiente enlace.

\begin{center}
    \url{https://youtu.be/m7rvcdZuUMI}
\end{center}


\section{Entrega 2}

La segunda entrega del proyecto estuvo más centrada en el modelado de la sala del gótico que la primera. Tras diseñar el boceto de la misma, que puede verse en la figura \ref{fig:bocetos-sala-2}, se modeló el Blender. Al ser un modelo lleno de detalles y diseños y formas muy específicos, como los arcos o las ventanas, se tardó bastante en terminar. El resultado final puede verse en la figura \ref{fig:unity-sala-2}.

\begin{figure}[!h]
\begin{center}
\includegraphics[width=0.75\textwidth]{imagenes/7/bocetos/boceto-sala-2.png}
\caption{Boceto de la segunda sala}
\label{fig:bocetos-sala-2}
\end{center}
\end{figure}

\begin{figure}[!h]
\begin{center}
\includegraphics[width=0.85\textwidth]{imagenes/7/salas-unity/unity-sala-2.png}
\caption{Sala 2 renderizada en Unity}
\label{fig:unity-sala-2}
\end{center}
\end{figure}

Una vez se contaba con el modelo, se pasó a implementar la prueba. Para esta sala se decidió esconder una palanca tras uno de los cuadros laterales que, al activarla, abriera en dos el \textit{Jardín de las Delicias}. Para ello, una vez modelado el hueco en la pared con la palanca detrás del cuadro elegido, se utilizó el script del framework \acs{VRTK} \texttt{VRTK\_ArificialRotator} y se integró con un script propio para dotar de dicha funcionalidad tanto a la palanca como a la puerta. A continuación, en la figura \ref{fig:lever-controller-inspector} puede verse un fragmento de este último script desde el inspector de Unity y cómo contiene al primero. Además, también puede verse cómo se han asignado el resto de elementos mencionados para que pueda acceder a ellos y modificarlos.

Por ejemplo, desde este script es posible definir el contenido del cuadro de texto que muestra si el cuadro está abierto o cerrado, su velocidad de apertura y la referencia a los dos lados del cuadro, que se utilizará para acceder a ellos y modificar su posición.

\begin{figure}[!h]
\begin{center}
\includegraphics[width=0.6\textwidth]{imagenes/7/lever-controller.png}
\caption{Script para controlar la palanca desde el inspector}
\label{fig:lever-controller-inspector}
\end{center}
\end{figure}

A continuación, en el listado \ref{lst:lever-controller} se presenta simplificado el script utilizado para abrir y cerrar la puerta, que se puede ver completo en el archivo \texttt{LeverController.cs}. Coom puede verse, todo se realiza dentro del método \texttt{Update()}, actualizando en cada frame la posición de los elementos utilizando el método \texttt{Lerp}, que implementa una función de interpolación lineal.

\begin{lstlisting}[caption=Fragmento del script para abrir y cerrar una puerta abatible con una palanca, label=lst:lever-controller]
public Transform doorLeft;
public Transform doorRight;
public float openSpeed;

private void Update()
{
    doorLeft.position = Vector3.Lerp(doorLeft.position,
                                      leftPosition,
                                      Time.deltaTime * openSpeed);
    
    doorRight.position = Vector3.Lerp(doorRight.position,
                                       rightPosition,
                                       Time.deltaTime * openSpeed);
    

}
\end{lstlisting}

Además, multiplicando la velocidad de apertura definida antes por el tiempo transcurrido desde el último frame ejecutado (obtenido con la función \texttt{Time.detaTime} podemos asegurar que el todos los ordenadores en los que se ejecute se abrirá a la misma velocidad, evitando así que el juego se ejecute más rápido en ordenadores más potentes.

Por otro lado, para permitir al jugador abrir y cerrar el cuadro que esconde la palanca como si fuera ventana se ha utilizado el script \texttt{VRTK\_Physics\_-} \texttt{Rotator} al que, indicándole un vector que actúa como bisagra y el ángulo que soporta, además de otros parámetros, es capaz de rotar el objeto que lo contiene.

Y por último, se comenzó a prototipar la siguiente sala, aunque como apenas se empezó en esta entrega su explicación se deja para la siguiente.

Por tanto, como resultado de esta entrega se generó una versión del proyecto que contenía la antesala y las dos primeras salas funcionalmente terminadas y un prototipo con la estructura básica de la tercera. El vídeo que enseña los avances puede verse a continuación.

\begin{center}
    \url{https://youtu.be/kfEnxP5dHU4}
\end{center}


\section{Entrega 3}

\begin{figure}[!h]
\begin{center}
\includegraphics[width=1\textwidth]{imagenes/7/bocetos/boceto-sala-3.png}
\caption{Boceto de la tercera sala}
\label{fig:bocetos-salas-3}
\end{center}
\end{figure}

\section{Entrega 4}

\begin{figure}[!h]
\begin{center}
\includegraphics[width=1\textwidth]{imagenes/7/bocetos/boceto-sala-4.png}
\caption{Boceto de la cuarta sala}
\label{fig:bocetos-salas-4}
\end{center}
\end{figure}

\begin{figure}[!h]
\begin{center}
\includegraphics[width=1\textwidth]{imagenes/7/bocetos/boceto-sala-5.png}
\caption{Boceto de la quinta sala}
\label{fig:bocetos-salas-5}
\end{center}
\end{figure}

\section{Entrega 5}

\begin{figure}[!h]
\begin{center}
\includegraphics[width=1\textwidth]{imagenes/7/bocetos/boceto-sala-6.png}
\caption{Boceto de la sexta sala}
\label{fig:bocetos-salas-6}
\end{center}
\end{figure}

\section{Entrega 6}


\section{Entrega 7}

\chapter{Conclusiones y Trabajo futuro}
\label{chap:conclusiones}

En este capítulo se detallarán las conclusiones obtenidas del desarrollo de este proyecto, y se valorará si los objetivos propuestos en el capítulo \ref{chap:introduccion} han sido cumplidos. Además, se propondrán posibles líneas de trabajo futuro para mejorar este proyecto o que tomar como referencia para una posible segunda entrega.

\section{Conclusiones}

Este \acs{TFM} ha simulado el ciclo de desarrollo de un videojuego real en el que se ha implementado un videojuego completo con tecnologías \acs{VR}. Este desarrollo se ha dividido entregas, al final de las cuales se ha generado un entregable que se ha documentado a través de vídeos periódicos.

Tomando como referencia los resultados de la evaluación con usuarios reales realizada al final, se puede afirmar que se han cumplido los objetivos planteados al principio del proyecto; crear una experiencia inmersiva en la que el jugador se sintiera gratificado al resolver los diferentes acertijos que se le proponen. Además, esta experiencia puede motivarles para visitar museos reales. Por otro lado, si el usuario de \MineRVa padece algún tipo de impedimento físico que le impida hacerlo, se le ofrece una manera de emularlo haciendo uso del potencial motivador de un videojuego.

Por otro lado, con respecto a los objetivos secundarios, de índole más personal, se considera que este proyecto ha servido como primera toma de contacto tanto al desarrollo con tecnologías \acs{VR} y a sus principales frameworks y librerías como a los principales procesos y metodologías de desarrollo de videojuegos, así como al documento \acs{GDD} y las evaluaciones con usuarios.

Además, el trabajo de investigación realizado para el capítulo \ref{chap:estado_arte} ha servido para entender cómo han nacido las tecnologías utilizadas para desarrollar este proyecto y la motivación de sus creadores, como Blender o \acs{VRTK}, y todos problemas que tuvieron que superar para lograr sus objetivos.

Gracias a la división de trabajo al inicio del proyecto, la planificación por entregas y la utilización de metodologías ágiles ha sido posible terminarlo en el tiempo establecido, evitando así tener que realizar una fase de \textit{crunch} en la etapa final. Este término, que desafortunadamente está actualmente de moda, hace referencia al período en el que desarrolladores de un proyecto, normalmente informático, se ven obligados a trabajar por encima de las horas establecidas. Estos período suelen darse al final de los proyectos y se utilizan como única manera para poder llegar a la fecha de entrega estipulada. En su lugar, ha sido posible compaginarlo con el máster y las prácticas y completarlo a tiempo.

El desarrollo de este proyecto me ha permitido afianzar los conocimientos obtenidos a lo largo de las distintas asignaturas estudiadas en el máster, además de recordar conceptos aprendidos a lo largo de la carrera relacionados con interfaces y sistemas interactivos y de tiempo real.

En definitiva, considero que este proyecto es el principio de mi carrera profesional, que espero poder dedicar al desarrollo de sistemas interactivos y en la que espero poner en práctica los conocimientos adquiridos tras la realización de este \acs{TFM}.


\section{Trabajo futuro}

A pesar de haber completado todos los objetivos propuestos inicialmente y haber aportado algunos nuevos a lo largo del desarrollo, hay algunas opciones que no se han llevado a cabo o bien por falta de tiempo o bien porque no se cree que actualmente aporten suficiente valor al producto en comparación con su carga de trabajo.

Además, a diferencia de otro tipo de proyectos informáticos, los videojuegos son especialmente susceptibles a, en lugar de mejorar una versión, reunir todas estas mejoras y desarrollar una nueva entrega, teóricamente mejor que la anterior, por lo que es especialmente interesante plantear un posible trabajo futuro en proyectos de esta naturaleza. 

A continuación se presentan las líneas de trabajo futuro más relevantes que podrían seguirse de este proyecto.

\begin{itemize}
    \item \textbf{Traducir a diferentes idiomas.} Uno de los objetivos de un videojuego es que sea jugado por el mayor número de personas posibles, y una de las mayores barreras de entrada, si no la que más, es el lenguaje. 
    
    Por ello, se ha añadido el esqueleto de lo que en un futuro podría ser el soporte multilenguaje. Para lograr esto, se ha añadido un selector en los script de gestión de los diálogos y las descripciones de los cuadros en el que se puede seleccionar un lenguaje, que por debajo simplemente de qué carpeta leer los archivos de texto. En la figura \ref{fig:languaje-selector} puede verse el caso de este último, que actualmente solo soporta el español.

    \begin{figure}[!h]
    \begin{center}
    \includegraphics[width=0.6\textwidth]{imagenes/8/languaje-selector.png}
    \caption{Script para cambiar el idioma desde el inspector}
    \label{fig:languaje-selector}
    \end{center}
    \end{figure}

    Siguiendo este enfoque, cada lenguaje tendría una carpeta dentro del directorio de diálogos y descripciones. Para completarlo, en el menú principal sería necesario incluir una sección desde la que el usuario pudiera seleccionar su lenguaje, y ésto actualizase el resto de scripts.
    
    De este modo sería suficiente con proporcionar a un traductor los archivos de texto del juego e incluirlos directamente dentro de su carpeta, una vez traducidos, sin necesidad de modificar el código fuente. Además, sería posible que las traducciones de los diálogos se dividieran en un número mayor o menor de archivos, ya que el sistema continua leyéndolos hasta que no encuentra más, lo que les proporcionaría mayor flexibilidad y les permitiría a los traductores adaptar estas traducciones de mejor manera.

    \item \textbf{Guardar y cargar partidas.} Quizá uno de las mayores ausencias de funcionalidad del juego es la habilidad para poder guardar la partida y poder continuarla en un futuro. Esto no ha sido una prioridad a lo largo del desarrollo por varios motivos, siendo el principal que en un juego relativamente corto (si tomamos como referencia el vídeo mostrando el juego terminado, aproximadamente media hora) y que la dificultad de una sala reside en saber qué hay que hacer para avanzar, y una vez que se ha completado es inmediato, se considera que no es algo esencial y que el su lugar se podría añadir en un futuro.
    
    Además, por la naturaleza del juego sería relativamente sencillo guardar una partida en disco, bastando simplemente con almacenar la sala en la que se quedó el jugador, lo que si quisiéramos simplificar aún más podría traducirse en un selector de salas en el menú principal.
    
    \item \textbf{Aumentar la rejugabilidad.} Como ya ha podido verse en el punto anterior y en los resultados de las evaluaciones con usuarios, el principal problema del juego es su falta de rejugabilidad; esto es, una vez que el usuario completa el juego y conoce la historia y las salas y sus pruebas no tiene ninguna motivación por volver a jugar. 
    
    Esto podría intentar evitarse sin desviarse excesivamente de la naturaleza del juego variando, por ejemplo, los cuadros que hay en cada sala. Sería algo relativamente sencillo, y simplemente sería necesario elegir aleatoriamente qué cuadros va a tener una sala en el momento de cargarla. No sería una generación procedural completa, ya que la sala en sí sería la misma siempre; simplemente se sustituirían unos cuadros por otros que habrían sido modelados e incluidos previamente.
    
    \item \textbf{Publicación.} El paso lógico de un videojuego terminado es ser publicado, ya sea gratuito o de pago, y la plataforma en la que lo hacen la mayoría de los juegos es Steam, aunque por cuestión de agenda no ha sido posible hacerlo. Aún así, la versión compilada del juego está disponible públicamente en la sección \textit{releases} del repositorio del proyecto o, de manera equivalente, siguiendo el siguiente enlace.
    
    \begin{center}
        \url{https://github.com/gomezportillo/mineRVa/releases}
    \end{center}
    
\end{itemize}



\chapter{Apéndices}
\label{chap:apendices}

\section{Apéndice 1: Guía de salas}
\label{anexo:guia-salas}

\drop{E}{  n} las páginas siguientes se adjunta la primera versión de la estructura de las salas del museo que fue propuesta al inicio del proyecto junto a su disposición y sus cuadros.

\includepdf[pages=-]{anexos/guia-salas.pdf}


\section{Apéndice 2: Game Design Document}
\label{anexo:gdd}

A continuación se adjunta el \acs{GDD} final de \textit{MineRVa}.

\newpage

\includepdf[pages=-]{anexos/MineRVa-GDD.pdf}


\section{Apéndice 3: Evaluación de satisfacción con usuarios}
\label{anexo:evaluacion-satisfaccion}

A continuación se adjuntan los cuestionarios resueltos de las evaluaciones de satisfacción de los usuarios.

\newpage

\includepdf[pages=-]{anexos/evaluacion_usuarios.pdf}

\newpage

\section{Agradecimientos}


Me gustaría dar por concluido este máster con unas pequeñas líneas en las que tener la oportunidad de recordar a todas las personas que han hecho posible de un modo u otro este \acs{TFM}.

Antes de nada, me gustaría agradecer al doctor Francisco Gutiérrez su orientación, motivación y tutela a lo largo del desarrollo de este proyecto.

También me gustaría recordar a todos los amigos que he conocido durante este año, tanto en clase como en las prácticas. Especialmente, a Juan Carlos Serrano y a Felipe Peiró; sin ellos, este curso no habría sido lo mismo.

Agradecer también a todos los amigos que han estado conmigo desde pequeño; a Álvaro, Jaime, Edu y a todos los demás. Y, por supuesto, a mi historiadora del arte personal.

Y por último, me gustaría dedicarles estas últimas líneas a mis padres, por todo lo que me han enseñado.

Gracias a todos.

\nocite{*}
\bibliography{bibliografia/bibliografia}
\addcontentsline{toc}{chapter}{Bibliografía}
\bibliographystyle{apalike}


\end{document}
