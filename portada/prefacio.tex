\chapter*{}

\begin{titlepage}
 
 
\setlength{\centeroffset}{-0.5\oddsidemargin}
\addtolength{\centeroffset}{0.5\evensidemargin}
\thispagestyle{empty}

\noindent\hspace*{\centeroffset}
\begin{minipage}{\textwidth}
\centering
\vspace{3.3cm}
%\includegraphics[width=3cm]{imagenes/logo} 
\vspace{0.5cm}

{\Huge\bfseries \myTitle \\}
\noindent\rule[-1ex]{\textwidth}{3pt}\\[3.5ex]
{\large\bfseries \mySubtitle \\[4cm]}
\end{minipage}

\vspace{2.5cm}

\noindent\hspace*{\centeroffset}
\begin{minipage}{\textwidth}
\centering
\textbf{Autor}\\ {\myName}\\[2.5ex]
\textbf{Director}\\ {\myProf}\\[2cm]
\end{minipage}

\vspace{\stretch{2}}

\end{titlepage}




\cleardoublepage
\thispagestyle{empty}

\copyrightpage

\begin{center}
{\large\bfseries \myTitle: \mySubtitle}\\
\end{center}

\begin{center}
\myName \\
\end{center}

\vspace{0.7cm}
\noindent{\textbf{Palabras clave}: \myPalabrasclave }\\

\vspace{0.7cm}
\noindent{\textbf{Resumen}}\\

% Resumen
los videojuegos permiten a sus usuarios sumergirse en mundos ficticios y meterse en la piel de personajes muy diferentes a ellos mismos en el mundo real, permitiéndoles vivir historias que de otro modo no serían capaces. Uno de los grandes cambios en en paradigma de los videojuegos está siendo la Realidad Virtual, que utilizan tecnologías inmersivas para aumentar en gran medida la sensación de realismo.

\bigskip

Específicamente, las tecnologías asociada a la Realidad Virtual permiten que personas que padece algún tipo de impedimento físico y que no pueden acceder a centros, como pueden ser museos, puedan visitarlos de una forma relativamente realista y en muchos casos incluso motivadora por la propia naturaleza de la tecnología que se emplea.Este proyecto tiene como objetivo desarrollar un sistema interactivo e inmersivo utilizando tecnologías de Realidad Virtual enmarcado en los procesos de desarrollo de un videojuego real y aplicando metodologías ágiles.

\bigskip

La temática de este sistema se basa en un museo en el que el jugador debe encontrar un cuadro robado ayudado por el vigilante a través de diversas salas correspondientes a las distintas etapas de la historia de arte, resolviendo pruebas temáticas en cada una.

\bigskip

Este proyecto explorará diversas mecánicas de interacción con el usuario y objetos virtuales, haciendo uso de físicas avanzadas e intentará integrarlas en su narrativa de la manera más orgánica posible para el jugador. 


\thispagestyle{empty}

\cleardoublepage

\begin{center}
	{\large\bfseries \myTitle: \myEnglishSubtitle}\\
\end{center}

\begin{center}
	\myName \\
\end{center}

\vspace{0.7cm}
\noindent{\textbf{Keywords}: \myKeywords }\\

\vspace{0.7cm}
\noindent{\textbf{Abstract}}\\

% Abstract




% \chapter*{}
% \thispagestyle{empty}

% \noindent\rule[-1ex]{\textwidth}{2pt}\\[4.5ex]

% Yo, \textbf{\myName}, alumno de la titulación \myDegree de la \textbf{\myFaculty}, con DNI 71722388Q, autorizo la ubicación de la siguiente copia de mi Trabajo Fin de Máster en la biblioteca del centro para que pueda ser consultada por las personas que lo deseen.

% \vspace{6cm}

% \noindent Fdo: \myName

% \vspace{2cm}

% \begin{flushright}
% \myLocation a \myTime.
% \end{flushright}


\chapter*{}
\thispagestyle{empty}

\noindent\rule[-1ex]{\textwidth}{2pt}\\[4.5ex]

D. \textbf{\myProf}, Profesor del \myDepartment de la \myUni.

\vspace{0.5cm}

\textbf{Informa:}

\vspace{0.5cm}

Que el presente trabajo, titulado \textit{\textbf{\myTitle, \mySubtitle}}, ha sido realizado bajo su supervisión por \textbf{\myName}, y autoriza la defensa de dicho trabajo ante el tribunal que corresponda.

\vspace{0.5cm}

Y para que conste, expide y firma el presente informe en \myLocation a \myTime.

\vspace{1cm}

\textbf{El director:}

\vspace{5cm}

\noindent \textbf{\myProf}


\dedication{A mi historiadora del arte}
