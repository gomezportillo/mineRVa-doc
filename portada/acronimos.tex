\chapter{Lista de acrónimos}
{
\small
\begin{acronym}[XXXXXXXX]

\acro{IDE}     {Integrated Development Environment}
\acro{TFM}     {Trabajo Final de Máster}
\acro{VR}      {Virtual Reality}
\acro{VRTK}    {Virtual Reality Toolkit}

\acro{CSS}     {Cascading Style Sheets}
\acro{FPS}     {Frames Per Second}
\acro{GPU}     {Graphical Processing Unit}
\acro{GUI}     {Graphical User Interface}  
\acro{HCI}     {Human-Computer Interaction}
\acro{HTML5}   {Hypertext Markup Language, version 5}
\acro{HTTP}    {Hypertext Transfer Protocol}  
\acro{JPEG}    {Joint Photographic Experts Group}
\acro{JS}      {JavaScript}
\acro{KISS}    {Keep it simple, stupid!} 
\acro{RTS}     {Real Time System}
\acro{RUP}     {Rational Unified Process}
\acro{SDK}     {Software Development Kit}
\acro{TCP}     {Transmission Control Protocol}  
\acro{UPD}     {Unified Process Development}

\end{acronym}
}


% \ac{OO}   la primera vez \acf, después \acs
% \acs{OO}  short: OO
% \acf{OO}  full : Object Oriented (OO)
% \acl{OO}  large: Object Oriented
% \acx{OO}         OO (Object Oriented)

% usa \Acro cuando no debe aparecer nunca expandido en el texto

% Local variables:
%   TeX-master: "main.tex"
% End:
