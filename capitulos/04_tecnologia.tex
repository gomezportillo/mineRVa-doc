\chapter{Tecnología a usar}
\label{chap:tecnologia}

\section{Recursos}

\subsection{Recursos hardware}

\begin{itemize}
 \item \textbf{Lenovo Explorer.} Heatset VR
 \item \textbf{Asus R510V.}
\end{itemize}

\subsection{Recursos software}

\subsubsection{Sistema Operativo}

Windows 10 Edutation 

\subsubsection{Principales librerías}

\begin{itemize}
    \item \textbf{SteamVR.}
    \item \textbf{Windows Mixed Reality for SteamVR.}
    \item \textbf{VRTK.}\footnote{\url{https://vrtoolkit.readme.io}} es un framework de desarrollo multilibrería desarrollado en solitario por \textit{TheStoneFox}\footnote{\url{https://github.com/thestonefox}} (no se ha conseguido encontrar su nombre verdadero), un hombre residente en Birmingham, Reino Unido.
\end{itemize}

\subsubsection{Herramientas de desarrollo}

\begin{itemize}
    \item \textbf{Unity3D} \texttt{C\#}
    \item \textbf{Blender} 2.79c opensource
    \item \textbf{Visual Studio Community} Desarrollado por Microsoft
    \item \textbf{Atom} Desarrollado por GitHub
    \item \textbf{Git} Herramienta de control de versiones
\end{itemize}

\subsubsection{Herramientas de documentación}

\begin{itemize}
    \item \textbf{\LaTeX.} Para la documentación final. Overleaf
    \item \textbf{LibreOffice Writer.} (v5.5) Para informes 
    \item \textbf{LibreOffice Draw.} (v5.5) Para generar diagramas vectoriales
    \item \textbf{LibreOffice Calc.} (v5.5) Para generar gráficas.
    \item \textbf{GIMP.} (v2.8.20)Para editar imágenes
    \item \textbf{Inkscape.} (v0.92)Para editar imágenes vectoriales.
\end{itemize}