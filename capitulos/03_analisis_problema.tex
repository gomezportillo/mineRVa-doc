\chapter{Análisis inicial del problema}
\label{chap:analisis_problema}

muy parecido a la entrega 0; presentar lo que se va a hacer

linkear también desde aqui el anexo 1 , pero hablar de lo que se ha hecho finalmente, no de la idea inicial

\section{Concepto inicial}

Historia de un jugador

\section{Narrativa}

\section{Descripción de salas}

Diagrama de salas

aunque inicialmente se planteó hacer bifurcaciones en los caminos que llevan a las salas, de manera que el jugador pudiera elegir o incluso que tuviera que ir para atrás para acceder a alguna, finalmente se ha decidido hacer un museo lineal, principalmente porque se espera que sea mas facil para los jugadores, y porque tiene mas sentido en el contexto de la ambientación temporal de las salas que vayan una después de otra

(Adjuntar renderizados de cada sala para hacerlo visual? o los bocetos? o ambas?)


\subsection{Sala 0. Introducción}

\subsection{Sala 1. Tutorial}

\subsection{Sala 2. Gótico-flamenco}

\subsection{Sala 3. Renacimiento}

\subsection{Sala 4. Barroco}

\subsection{Sala 5. Romanticismo}

\subsection{Sala 6. Vanguardias del siglo XX}

\subsection{Sala 7. Final del juego}


