\chapter{Análisis inicial del problema}
\label{chap:analisis_problema}

En este capítulo se presenta el nacimiento y la idea inicial de la que parte este proyecto, así como su narrativa y la descripción del mundo virtual en el que tiene lugar.

\section{Concepto inicial}

Lo primero que se hizo antes de empezar a trabajar en el proyecto fue proponer una narrativa en la que basar el diseño y desarrollo; por ello, el primer paso que se dio en el proyecto fue redactar el documento \textbf{Guía de salas}, que puede verse en el apéndice \ref{anexo:guia-salas}. En este documento se redactó la primera versión de la narrativa del juego además de una propuesta con las salas que habría en el museo y qué obras de arte y puzzles tendría cada una. Este documento  fue enviado al tutor de este \acs{TFM} y, tras él dar el visto bueno al mismo, se comenzó a trabajar en el proyecto

Además, como se está trabajando siguiendo una metodología enmarcada en el desarrollo de videojuegos, como se explica en profundidad en el capítulo \ref{chap:metodologias}, el siguiente documento en el que se empezó a trabajar y que sirvió de andamio para el proyecto fue el \textbf{\acl{GDD}}, o \acs{GDD}. En un desarrollo real, este documento contendría la información necesaria para que todos sus integrantes tuvieran clara la idea global del juego, y en este caso es lo mismo; el documento contiene, entre otra información, el resumen, los objetivos o la narrativa del juego, y puede consultarse en el anexo \ref{anexo:gdd}.

En resumen, \MineRVa es un juego con una historia de un jugador desarrollado en un museo compuesto por varias habitaciones ambientadas en las principales épocas históricas con diferentes pruebas que el jugador tiene que superar para avanzar.

\section{Narrativa}

La narrativa del proyecto ha intentado ser enmarcada en un museo creíble y realista, de tal manera que se ha colocado al jugador en la piel de un detective infalible que ha sido contratado por el museo para resolver el misterio de la desaparición de su mayor obra de arte, y todo apunta a que ha sido robado. Este museo es el más grande del mundo y tiene obras de arte de todos los rincones, desde cuadros a estatuas.

El encargado de recibir al jugador en el museo es su vigilante, un vetusto y campechano señor que ha pasado toda su vida trabajando en él. Este personaje será el encargado de introducir al jugador la historia y las mecánicas del juego, de explicarle su misión y de guiarle a través de las distintas salas, ayudándole para que no se quede atascado.

Pero cuál es la sorpresa del jugador cuando, tras completar todas las pruebas y llegar al final del museo, descubre el misterioso ladrón de guante blanco es en realidad el propio vigilante, quien ha robado el cuadro para su disfrute personal.

Como puede verse, es necesario el vigilante disponga de un método para poder comunicar información al jugador, y lo que más encaja con la naturaleza de este proyecto son \textbf{diálogos unidireccionales}, ya que el jugador no podría hablar con él. El vigilante utilizará estos diálogos para explicar al usuario su misión en cada sala, le informará de algunos de sus errores y le felicitará cuando realice correctamente las tareas más importantes.

\section{Mundo virtual}

A continuación se describirán la estética y la composición del mundo virtual de este proyecto.

\subsection{Estética}

Como el presupuesto de este proyecto es nulo y el tiempo disponible para completarlo es especialmente bajo, la estética deberá ser acorde a estos factores. Por ello, se ha elegido una \textbf{estética low poly}, con lo que los tiempos de modelado se verán muy reducidos. Aún así, se intentará que no sea especialmente plana con la ayuda de texturas medianamente realistas que ayuden a los jugadores a conseguir una mayor inmersión y a dar mayor personalidad a las salas, pero todas ellas deberán ser gratuitas.

Por otro lado, aunque el modelado de las salas deberá ser hecho manualmente, se podrán importar de internet modelos tridimensionales especialmente complejos, como las estatuas que se crean necesarias para las salas; estos modelos suelen ser digitalizaciones exactas, que se adaptarán para reducir el número de polígonos, reducir el peso total del juego y hacer que no sea necesario un equipo hardware potente para jugarlo.

\subsection{Descripción de las salas}

Cuando se habla de salas en este proyecto se refiere a las unidades que presentan una historia, una temática y un puzzle distinto a las demás. El museo estará compuesto por 7 salas, aparte de la de tutorial, y cada una de ellas hará referencia a un período artístico distinto y estará ambientada  decorada de manera coherente con él.

Aunque inicialmente se planteó hacer bifurcaciones en los caminos que llevan a las salas de manera que el jugador pudiera elegir o incluso que tuviera que ir para atrás para acceder a alguna tras desbloquearla, finalmente se ha decidido hacer un museo lineal, principalmente porque se espera que sea más fácil para los jugadores, y porque tiene más sentido en el contexto de la ambientación temporal de las salas que vayan una después de otra.

El objetivo de cada nivel o sala será resolver el puzzle propuesto para acceder a la siguiente, además de que el usuario interactúe con las obras de arte y aprenda sobre ellas, ya que en alguno de los casos le será necesario.

Como se ha indicado antes, en total habrá \textbf{8 salas}. La primera a la que accede el jugador será la del \textbf{tutorial}, que tendrá un puzzle especialmente fácil para que el jugador se pueda centrar en entender el concepto, las mecánicas y los controles del juego.

Tras ella el jugador accederá a la sala corresponderá al período \textbf{gótico-flamenco}, el primero desde el que se ha decidido trabajar y estará ambientado en el interior de una iglesia gótica, con paredes de piedra altas y bóvedas de crucería. Además, tendrá tres cuadros; a los lados, \textit{El Matrimonio Arnolfini} (1434)  de Jan van Eycky \textit{El descendimiento} de van der Weyden (1436), a modo representativo del género, y en la pared del fondo \textit{El jardín de las delicias} de El Bosco (alrededor del 1500), que se abrirá por la mitad para mostrar la puerta al siguiente nivel

La siguiente sala estará ambientada en el \textbf{renacimiento}, y tendrá ventanas amplias y frisos de mármol. Los cuadros de esta sala serán \textit{La última cena} de Leonardo da Vinci (alrededor de 1495), \textit{La escuela de atenas} de Rafael (1511) y \textit{La creación de Adán}, de Miguel Ángel Buonarroti (1511). Para ambientar la sala también se utilizarán esculturas 3D, como \textit{Piedad} o \textit{Moisés}, ambas de Miguel Ángel, ya modeladas y con licencias de uso libre.

A continuación se encontrará la sala del \textbf{barroco}, con altas paredes de piedra. El concepto de esta sala ha cambiado varias veces a lo largo del desarrollo del proyecto, y por cuadros finalmente se han utilizado cuatro cuadros de San Sebastián, que fue un santo al que asetearon hasta la muerte y del que se hicieron muchas versiones a lo largo de este período. Los pintores elegidos han sido será de Guido Reni (1625),  José de Ribera (1636), Mattia Preti (1657) y Peter Paul Rubens (1614).

La siguiente sala corresponderá al \textbf{romanticismo}, por lo que se habrá poca iluminación y se utilizarán texturas de madera y un sistema de partículas de niebla para crear inmersividad. Los cuadros elegidos han sido \textit{Fusilamiento de Torrijos y sus compañeros en las playas de Málaga} de Antonio Gisbert Pérez (1887), \textit{Doña Juana la Loca} de Francisco Pradilla y Ortiz (1877) y \textit{El caminante sobre un mar de nubes} de Caspar David Friedrich (1818).

La última sala estará ambientada en las \textbf{vanguardias del siglo XX}. Aunque en este período coexisten multitud de estilos distintos, se han elegido el \textbf{cubismo}, el \textbf{expresionismo}, el \textbf{surrealismo} y el \textbf{postimpresionismo} como los más representativos, por lo que los cuadros utilizados para esta sala son, respectivamente, el \textit{Guernica} de Pablo Picasso (1937), \textit{El grito} de Edvard Munch (1910), \textit{La tentación de San Antonio} de Salvador Dalí (1946) y \textit{La noche estrellada} de Vincent van Gogh (1889).

Cuando el jugador complete todas las salas podrá acceder a la \textbf{sala final}, en la que descubrirá que el vigilante que creía su amigo es en realidad el ladrón de guante blanco al que ha estado persiguiendo desde el principio.

\subsection{Retos}

Como ha podido verse, las salas que componen el museo son muy variadas entre sí, y los acertijos en ellas también. Para evitar que las salas sean monótonas y repetitivas, cada una de ellas tendrá un reto totalmente diferente y hecho a medida que el jugador tendrá que superar para acceder a la siguiente sala. 

Todos estas pruebas estarán enmarcadas dentro de diversas dinámicas de Realidad Virtual, donde el jugador deberá ser proactivo e investigar qué debe hacer y cómo hacerlo. Por ejemplo, deberá observar los cuadros para encontrar mensajes ocultos, interactuar con ellos y con otros elementos del entorno o incluso hacer que los propios cuadros cambien para poder avanzar.